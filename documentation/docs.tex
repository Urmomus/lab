%!TEX TS-program = xelatex
\documentclass[a4paper]{article}
\usepackage[a4paper, mag=1000, left=2.5cm, right=1cm, top=2cm, bottom=2cm, headsep=0.7cm, footskip=1cm]{geometry}
\usepackage[utf8]{inputenc}
\usepackage[english,russian]{babel}
\usepackage{multirow}
\usepackage{graphicx}
\usepackage{listings}
%\usepackage[colorlinks]{hyperref}

\usepackage{enumitem}

\usepackage{fontspec}
\setmainfont{Times New Roman}
\setsansfont{Arial}
\setmonofont{Courier New}
\newfontfamily\cyrillicfont[Script=Cyrillic]{Times New Roman}
\newfontfamily\cyrillicfontsf[Script=Cyrillic]{Arial}
\newfontfamily\cyrillicfonttt[Script=Cyrillic]{Courier New}
\usepackage{polyglossia}
\setdefaultlanguage{russian}
\usepackage{listings}

\lstdefinestyle{CppCodeStyle}{
	basicstyle=\footnotesize\ttfamily,
	language={[ANSI]C++},
	keywordstyle=\bfseries,
	showstringspaces=false,
	morekeywords={include, printf},
	commentstyle={},
	escapeinside=§§,
	escapebegin=\begin{russian}\commentfont,
		escapeend=\end{russian},
}
\newcommand{\commentfont}{\ttfamily}

\begin{document}
	\def\contentsname{Содержание}
	
	% Оформление титульного листа
	\begin{titlepage}
		\begin{center}
			\textbf{Московский Авиационный Институт\\[10mm]}
				Институт №3 \\
				Программная инженерия \\
				Кафедра 304 \\
			
			\vfill
			
			\textbf{Отчёт о лабораторной работе \\ 
				по учебной дисциплине "Информационные технологии" \\
				на тему \\
				"Вычисление суммы бесконечного ряда" \\ [50mm]
			}
		\end{center}
		
		\hfill
		\begin{minipage}{.5\textwidth}
			Выполнили: студенты группы М3О-111Б-22 \\ [2mm] 
			Яковченко Н.Р.
			
			Деккер С.А. \\ [3mm]
			
			Приняли: \\ [2mm]
			Секретарёв В.Е.
			
			Давыдкина Е.А.
			
		\end{minipage}%
		\vfill
		\begin{center}
			Москва, \the\year\ г.
		\end{center}
	\end{titlepage}
	
	% Содержание
	\tableofcontents
	\newpage
	
	\section{Задание}
	
	\begin{center}
	\rule{\textwidth}{1pt} \\[2mm]
		\begin{tabular}{ lllll }
			\multicolumn{1}{l}{Кафедра 304} & \multicolumn{4}{r}{Курс: ИНФОРМАЦИОННЫЕ ТЕХНОЛОГИИ} \\[2mm]
			\multicolumn{3}{l}{Задание: \textbf{Вычисление суммы бесконечного числового ряда}} & & \\[2mm]
		\end{tabular}
	\rule{\textwidth}{1pt} \\[2mm]
	Вариант №11
	\end{center}
	
	Определить с заданной точностью сумму членов бесконечного степенного ряда:
	
	$$ 1 + \displaystyle\sum_{n=1} \frac{x^{2n}}{(2n)!} = 1 + \frac{x^{2}}{2!} + \frac{x^{4}}{4!} + \frac{x^{6}}{6!} + ... $$
	
	Предусмотреть ввод точности и печать: количества просуммированных элементов, суммы, разности суммы и точного значения, которое равно:
	
	$$ \frac{e^{x} + e^{-x}}{2} = ch(x) $$
	
	\newpage
	
	\section{Схема алгоритма}
	
	\begin{center}
		\includegraphics[width=\textwidth]{printed.pdf}
	\end{center}
	
	\newpage
	
	\section{Текст программы}
	
	\
	\begin{lstlisting}[style={CppCodeStyle}]
	/************************************************************
 	 *                                                          *
 	 *           §Информатика и вычислительная техника§           *
 	 *                                                          *
 	 ************************************************************
 	 *  Project type: ConsoleApplication                        *
 	 *  Project name: lab                                       *
 	 *  Language: cpp gcc version 9.4.0                         *
	 *  Programmers: §М3О-111Б-22§                                *
	 *               §Яковченко Николай Романович;§               *
	 *               §Деккер Сергей Альбертович§                  *
	 *  Modigied by:                                            *
	 *  Created: 18.10.2022                                     *
	 *  Comment: §FirstProgramm (Подсчёт суммы ряда)§             *
	 ************************************************************/
		
		
	// §Подключение библиотек§
	#include <iostream>
	#include <cmath>
	using namespace std;
	
	
	int main() // §Начало программы§
	{
		setlocale(LC_ALL, "rus"); // §Подключение русского языка§
		
		int n;                    // §Число слагаемых§
		unsigned long long doubleFactorial; // §Текущее значение удвоенного факториала§
		double difference;        // §Разность текущей и точной сумм§
		double currentSum;        // §Сумма§
		double exactSum;          // §Точное значение суммы§
		double x;                 // §Переменная§
		double eps;               // §Точность§
		double numerator;         // §Числитель§
		
		cout << "Введите точность: ";
		cin >> eps;               // §Ввод точности§
		cout << eps << endl;      // §Эхо-печать§
		
		if (eps <= 0)             // §Валидация входящих данных§
		{
			// §Вывод сообщения об ошибке§
			cout << "§Заданная точность должна быть больше нуля§" << endl; 
			// §Завершение работы программы в случае некорректности введённых данных§
			return 1;
		}
		
		cout << "Введите X: ";
		cin >> x;                 // §Ввод переменной x§
		cout << x << endl;        // §Эхо-печать§
		
		// §Инициализация переменных§
		n = 1;
		doubleFactorial = 1;
		currentSum = 1;
		numerator = 1;
		
		exactSum = (exp(x) + exp(-x)) / 2;       §// Подсчёт точной суммы§
		difference = abs(currentSum - exactSum); §// Подсчёт разности§
		
		while (difference > eps)  // §Начало цикла§
		{
			// §Вычисление суммы ряда§
			doubleFactorial = doubleFactorial * (2 * n) * (2 * n - 1);
			numerator = numerator * x * x;
			currentSum += numerator / doubleFactorial;
			n += 1;
			difference = abs(currentSum - exactSum);
		} // §Конец цикла§
		
		// §Вывод значений переменных§
		cout << "§Сумма членов: §" << currentSum << endl;
		cout << "§Точное значение: §" << exactSum << endl;
		cout << "§Число слагаемых: §" << n << endl;
		cout << "§Разность суммы и точного значения: §" << difference << endl;
		return 0; // §Возврат значения 0§
	} // §Конец программы§
	
	\end{lstlisting}
	
	\section{Тесты}
	\subsection{Некорректные тесты}
	
	\begin{enumerate}[label=\textbf{Тест \arabic*}]
		\item Цель: проверить работу программы на границе некорректной области \\
		Исходные данные: eps = 0 \\
		Ожидаемый результат: Заданная точность должна быть больше нуля \\
		Полученный результат:
		
		\begin{figure}[h]
			\includegraphics[width=\textwidth,trim=0.5mm 0 0 0.5mm,clip]{tests/test0.png}
		\end{figure}
	
		Вывод по тесту: Полученный результат совпал с ожидаемым. Тест ошибок не выявил.
		\vspace{5mm}
		
		\item Цель: проверить работу программы на границе некорректной области \\
		Исходные данные: eps = -1.02 \\
		Ожидаемый результат: Заданная точность должна быть больше нуля \\
		Полученный результат:
		
		\begin{figure}[h]
			\includegraphics[width=\textwidth,trim=0.5mm 0 0 0.5mm,clip]{tests/test-1.02.png}
		\end{figure}
	
		Вывод по тесту: Полученный результат совпал с ожидаемым. Тест ошибок не выявил.
	\end{enumerate}
	
	\subsection{Корректные тесты}
	\renewcommand{\arraystretch}{1.5}
	\begin{enumerate}[label=\textbf{Тест \arabic*},start=3]
		\item Цель: проверить работу цикла по расчёту суммы ряда при высокой погрешности \\
		Исходные данные: eps = 100, x = 1 \\
		
		\begin{tabular}{l l}
			Ожидаемый результат: & Сумма членов: 1                           \\
			                     & Точное значение: 1,5308                   \\
			                     & Число слагаемых: 1                        \\
			                     & Разность суммы и точного значения: 0,5308 \\[4mm]
		\end{tabular}
	
		Полученный результат:
		
		\begin{figure}[h]
			\includegraphics[width=\textwidth,trim=0.5mm 0 0 0.5mm,clip]{tests/test100.png}
		\end{figure}
		
		Вывод по тесту: Полученный результат совпал с ожидаемым. Тест ошибок не выявил.
		\newpage
		
		\item Цель: проверить работу цикла по расчёту суммы ряда \\
		Исходные данные: eps = 0.001, x = 3 \\
		
		\begin{tabular}{l l}
			Ожидаемый результат: & Сумма членов: 10,0676                                \\
			                     & Точное значение: 10,0677                             \\
			                     & Число слагаемых: 7                                   \\
			                     & Разность суммы и точного значения: $\approx 0,000057$ \\[4mm]
		\end{tabular}
		
%		\begin{tabular}{l|l|l|l}
%			n & currentSum & exactSum                  & difference \\
%			1 & 1          & \multirow{7}{*}{10,06766} & 9,06766    \\
%			2 & 5,5        &                           & 4,56766    \\
%			3 & 8,875      &                           & 1,19266    \\
%			4 & 9,8875     &                           & 0,18016    \\
%			5 & 10,05022   &                           & 0,01744    \\
%			6 & 10,0665    &                           & 0,00116    \\
%			7 & 10,0676    &                           & 0,00006
%		\end{tabular}
		\begin{tabular}{l|l|l|l}
			n & currentSum                                           & exactSum                    & difference                           \\
			1 & $ 1 $                                                & \multirow{7}{*}{$10,06766$} & $ 10,067662 - 1 = 9,067662 $         \\
			2 & $ 1 + \frac{9}{2} = 5,5 $                            &                             & $ 10,067662 - 5,5 = 4,567662 $       \\
			3 & $ 5,5 + \frac{81}{24} = 8,875 $                      &                             & $ 10,067662 - 8,875 = 1,192662 $     \\
			4 & $ 8,875 + \frac{729}{720} = 9,8875 $                 &                             & $ 10,067662 - 9,8875 = 0,180162 $    \\
			5 & $ 9,887500 + \frac{6561}{40320} = 10,050223 $        &                             & $ 10,067662 - 10,050223 = 0,017439 $ \\
			6 & $ 10,050223 + \frac{59049}{3628800} = 10,066496 $    &                             & $ 10,067662 - 10,066496 = 0,001166 $ \\
			7 & $ 10,066496 + \frac{531441}{479001600} = 10,067605 $ &                             & $ 10,067662 - 10,067605 = 0,000057 $
		\end{tabular}
		
		Полученный результат:
		
		\begin{figure}[h]
			\includegraphics[width=\textwidth,trim=0.5mm 0 0 0.5mm,clip]{tests/test0.001.png}
		\end{figure}
		
		Вывод по тесту: Полученный результат совпал с ожидаемым. Тест ошибок не выявил.
		\newpage
		
		\item Цель: проверить работу цикла по расчёту суммы ряда \\
		Исходные данные: eps = 50, x = 10 \\
		
		\begin{tabular}{l l}
			Ожидаемый результат: & Сумма членов: 11002,4                             \\
			                     & Точное значение: 11013.2                          \\
			                     & Число слагаемых: 11                               \\
			                     & Разность суммы и точного значения: 10,7935 \\[4mm]
		\end{tabular}
	
		\hspace{-1cm}
		\begin{tabular}{l|l|l|l}
			n  & currentSum                                                                      & exactSum                     & difference                             \\
			1  & $ 1 $                                                                           & \multirow{11}{*}{11013,2329} & $ 11013,2329 - 1 = 11012,2329 $        \\
			2  & $ 1 + \frac{100}{2} = 51 $                                                      &                              & $ 11013,2329 - 51 = 10962,2329 $       \\
			3  & $ 51 + \frac{10000}{24} = 467,6667 $                                            &                              & $ 11013,2329 - 467,6667 = 10545,5663 $ \\
			4  & $ 467,6667 + \frac{1000000}{720} = 1856,5556 $                                  &                              & $ 11013,2329 - 1856,5556 = 9156,6774 $ \\
			5  & $ 1856,5556 + \frac{100000000}{40320} = 4336,7143 $                             &                              & $ 11013,2329 - 4336,7143 = 6676,5186 $ \\
			6  & $ 4336,7143 + \frac{10000000000}{3628800} = 7092,4462 $                         &                              & $ 11013,2329 - 7092,4462 = 3920,7867 $ \\
			7  & $ 7092,4462 + \frac{1000000000000}{479001600} = 9180,1219 $                     &                              & $ 11013,2329 - 9180,1219 = 1833,1110 $ \\
			8  & $ 9180,1219 + \frac{100000000000000}{87178291200} = 10327,1965 $                &                              & $ 11013,2329 - 10327,1965 = 686,0365 $ \\
			9  & $ 10327,1965 + \frac{10000000000000000}{20922789888000} = 10805,1442 $          &                              & $ 11013,2329 - 10805,1442 = 208,0887 $ \\
			10 & $ 10805,1442 + \frac{1000000000000000000}{6402373705728000} = 10961,3363 $      &                              & $ 11013,2329 - 10961,3363 = 51,8967 $  \\
			11 & $ 10961,3363 + \frac{100000000000000000000}{2432902008176640000} = 11002,4394 $ &                              & $ 11013,2329 - 11002,4394 = 10,7935 $
		\end{tabular}
		
		%\begin{tabular}{l|l|l|l}
	%		n  & currentSum & exactSum                  & difference \\
	%		1  & 1          & \multirow{11}{*}{11013.2} & 11012.2    \\
	%		2  & 51         &                           & 10962,2    \\
	%		3  & 467,6      &                           & 10545,6    \\
	%		4  & 1856,5     &                           & 9156,7     \\
	%		5  & 4436,7     &                           & 6576,5     \\
	%		6  & 7092,4     &                           & 3920,8     \\
	%		7  & 9180,1     &                           & 1833,1     \\
	%		8  & 10327,19   &                           & 686,01     \\
	%		9  & 10805,1    &                           & 208,1      \\
	%		10 & 10961,3    &                           & 51,9       \\
	%		11 & 11002,4    &                           & 10,8
	%	\end{tabular}
		
		
		Полученный результат:
		
		\begin{figure}[h]
			\includegraphics[width=\textwidth,trim=0.5mm 0 0 0.5mm,clip]{tests/test50.png}
		\end{figure}
		
		Вывод по тесту: Полученный результат совпал с ожидаемым. Тест ошибок не выявил.
	\end{enumerate}
	
	\section{Вывод}
	
	Разработка программы завершена на том основании, что:
	
	\begin{enumerate}
		\item Все полученные результаты совпали с ожидаемыми
		\item Мы считаем набор тестов полным
	\end{enumerate}
\end{document}